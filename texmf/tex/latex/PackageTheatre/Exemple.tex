%%%exemple pour utiliser PackageTheatre.sty
\documentclass[11pt, a4paper]{report}
%Astuce : \documentclass[11pt, a4paper, twoside]{report} pour différencier les pages de gauche des pages de droite.
\usepackage{PackageTheatre}
\usepackage[top=25mm,bottom=25mm,left=15mm,right=15mm]{geometry}



\newcommand{\pt}{\textit{PackageTheatre.sty}}

\RefCoinPage%Permet d'avoir l'acte et/ou la scène en cours en haut de page.
\auteurpiece{L. \textsc{Ducobu}}%Permet d'avoir le nom de l'auteur de la pièce en haut de la page
%\InverseSensRef%Permet d'inverser le sens des références (Inverser la position du nom de l'auteur et de la scène en cours)
%\EffaceRefScene%Permet de ne pas avoir l'acte et/ou la scène en cours en haut de la page (affiuche alors le titre par défaut)
%\EffaceRefTitre%Permet de ne pas avoir le titre affiché en haut de la page dans le cas où celui-ci est affiché (voir ci dessus)
\ActiveEspaceScene%Permet d'aérer le texte en entamant une nouvelle page au début d'une scène
%\ArreteEspaceScene%Permet d'annuler la séparation des scènes à partir d'une scène donnée jusqu'à l'emploi éventuel d'un nouveau ``\ActiveEspaceScene''
%\EffaceActeTableMatiere%Permet de n'afficher que le titre de la pièce dans la table des matières
%\EffaceSceneTableMatiere%Permet de n'afficher que le titre de la pièce et les actes (si présents) dans la table des matières [mais pas les différentes scènes]

\begin{document}
	\title{Exemple d'utilisation de ``\textit{PackageTheate.sty}''}
	\author{Ludovic \textsc{Ducobu}}
	\date{}
	\maketitle
	\tableofcontents
	
	\section{Situation}
	
	Écrire un pièce avec \LaTeX \ c'est super facile avec ``\pt'' !
	
	\section{Distribution des rôles}
	
	Marc
	\vspace{2mm}
	
	Mathieu:
	\vspace{2mm}
	
	Luc:
	\vspace{2mm}
	
	Jean:
	\vspace{2mm}
	
	\section{Notes sur les personnages}
	
	Ils sont relativement inintéressants...
	
	\CreePersonnage{Marc}{Marc}
	\CreePersonnage{Mathieu}{Mathieu}
	\CreePersonnage{Luc}{Luc}
	\CreePersonnage{Jean}{Jean}
	\CreePersonnage{tous}{tous}
	
	\begin{piece}
		\titrepiece{La nuit tous les chats sont gris}
		
		\begin{acte}
			\scene
			
			\didascalie{À l'ouverture du rideau, allumer la lumière... Sinon les gens ne verront rien...}
			
			\Marc{\didascalie{Sur un ton maussade :} Bonjour.}
			\Mathieu{Salut!}
			\Marc{Beau temps aujourd'hui...}
			\scene
			\Jean{C'est l'appocalypse! l'appocalyyyyypse !!!}
			\tous{La ferme, Jean !!!}
		\end{acte}
		
		\begin{acte}
			\scene
			\Luc{\didascalie{manifestement éméché :} Alors c'est satan qui rentre dans un bar...}
			
			\didascalie{Censuré}
			
			\scene
			\ArreteEspaceScene
			\Marc{Tout ça ne peut rien donner de bon...}
			\Jean{Lorem ipsum dolor sit amet, consectetur adipisicing elit, sed do eiusmod tempor incididunt ut labore et dolore magna aliqua. Ut enim ad minim veniam, quis nostrud exercitation ullamco laboris nisi ut aliquip ex ea commodo consequat. Duis aute irure dolor in reprehenderit in voluptate velit esse cillum dolore eu fugiat nulla pariatur. Excepteur sint occaecat cupidatat non proident, sunt in culpa qui officia deserunt mollit anim id est laborum.
			\\
			Sed ut perspiciatis unde omnis iste natus error sit voluptatem accusantium doloremque laudantium, totam rem aperiam, eaque ipsa quae ab illo inventore veritatis et quasi architecto beatae vitae dicta sunt explicabo. Nemo enim ipsam voluptatem quia voluptas sit aspernatur aut odit aut fugit, sed quia consequuntur magni dolores eos qui ratione voluptatem sequi nesciunt. Neque porro quisquam est, qui dolorem ipsum quia dolor sit amet, consectetur, adipisci velit, sed quia non numquam eius modi tempora incidunt ut labore et dolore magnam aliquam quaerat voluptatem. Ut enim ad minima veniam, quis nostrum exercitationem ullam corporis suscipit laboriosam, nisi ut aliquid ex ea commodi consequatur? Quis autem vel eum iure reprehenderit qui in ea voluptate velit esse quam nihil molestiae consequatur, vel illum qui dolorem eum fugiat quo voluptas nulla pariatur?
			\\
			At vero eos et accusamus et iusto odio dignissimos ducimus qui blanditiis praesentium voluptatum deleniti atque corrupti quos dolores et quas molestias excepturi sint occaecati cupiditate non provident, similique sunt in culpa qui officia deserunt mollitia animi, id est laborum et dolorum fuga. Et harum quidem rerum facilis est et expedita distinctio. Nam libero tempore, cum soluta nobis est eligendi optio cumque nihil impedit quo minus id quod maxime placeat facere possimus, omnis voluptas assumenda est, omnis dolor repellendus. Temporibus autem quibusdam et aut officiis debitis aut rerum necessitatibus saepe eveniet ut et voluptates repudiandae sint et molestiae non recusandae. Itaque earum rerum hic tenetur a sapiente delectus, ut aut reiciendis voluptatibus maiores alias consequatur aut perferendis doloribus asperiores repellat.}
			\tous{La ferme, Jean !!!}
			\scene
			
			\didascalie{Une scène palpitante que je n'ai pas eu le temps ni le talent d'écrire...}
			
		\end{acte}
		
	\end{piece}
	
	\auteurpiece{un anonyme honteux}
	\section{Pour reprendre son souffle}
	
	Une section pour se remettre de ses émotions...
	
	Avec le compte des répliques de chaque personnage !
	
	\begin{center}
		\begin{tabular}{|c|c|}
		\hline Personnage & Nombre de répliques \\
		\hline \Marc & \theMarc \\
		\hline \Mathieu & \theMathieu \\
		\hline \Luc & \theLuc \\
		\hline \Jean & \theJean \\
		\hline \tous & \thetous \\
		\hline
	\end{tabular}
	
	Nombre total de répliques : \thetotalrepliques
	\end{center}
	
	%Pour remettre à zéro le compteur des répliques de chaque personnage.
	%Essayer de commenter ces lignes pour voir la différence sur le compte des répliques de chaque personnage. 
	\CreePersonnage{Marc}{Marc}
	\CreePersonnage{Mathieu}{Mathieu}
	\CreePersonnage{Luc}{Luc}
	\CreePersonnage{Jean}{Jean}
	\CreePersonnage{tous}{tous}
	
	\EffaceRefScene
	\begin{piece}
		\titrepiece{La nuit tous les chats sont gris 2 : Le retour}
		
		\didascalie{À l'ouverture du rideau... Surtout ne pas allumer la lumière !!! \MakeUppercase{ç}a permettra aux comédiens de fuir...}
		\scene
		
		\Jean{Moi je me casse !}
		\Marc{Ta geu...}
		\Luc{Non attends...}
		\ActiveEspaceScene
		\scene
		\Mathieu{Il a raison pour le coup}
		\tous{Moi je me casse !}
		\ArreteEspaceScene
		\scene
		
		\didascalie{
			Y a plus de comédien sur scène... Ni \Mathieu, ni \Marc, ni \Luc, ni \Jean \dots}
		
		\begin{center}
			\vspace{15mm}
			\didascalie{noir}
			
			\vspace{10mm}
			{\Huge FIN}
		\end{center}
	\end{piece}
	
	\auteurpiece{Personne}
	\section{Décompte des répliques dans la seconde pièce}
	\begin{center}
		\begin{tabular}{|c|c|}
			\hline Personnage & Nombre de répliques \\
			\hline \Marc & \theMarc \\
			\hline \Mathieu & \theMathieu \\
			\hline \Luc & \theLuc \\
			\hline \Jean & \theJean \\
			\hline \tous & \thetous \\
			\hline
		\end{tabular}
		
		Nombre total de répliques : \thetotalrepliques
	\end{center}
	\section{Le mot de la fin}
	
	Au revoir !
	
	
	
	
	
\end{document}